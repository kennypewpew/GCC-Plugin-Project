\documentclass{beamer}
\usetheme{Berkeley}

\usepackage{algorithm}
\usepackage{algorithmic}
\usepackage{multicol}

\begin{document}

\section{Gestion des Directives}

\begin{frame}
  \frametitle{Pragmas}
  Nouveau pragma implement\'e
  \begin{block}
    {Utilisation}
    \#pragma MIHPS vcheck \space f1, f2, f3\\
    \#pragma MIHPS vcheck (f1, f2, f3)
  \end{block}
  \begin{itemize}
    \item Stock les noms de fonctions donn\'e par l'utilisateur
    \item Enleve les doublons
    \item Verifie que toutes fonctions ont \'et\'e trouv\'ees
  \end{itemize}
\end{frame}




\section{Instrumentation Dynamique des Boucles}

\begin{frame}
  \frametitle{Compilation - Analyse d'une Fonction}
  \begin{itemize}
    \item Verifie si la fonction est nomm\'e
    \item Parcour tout les boucles pour trouver les plus internes (loop$\rightarrow$inner = 0)
    \item Insert une initialisation pour pouvoir stocker les infos
    \item Parcour tout les basic\_blocks d'une boucle interne\\
      - Insert les fonctions de stockage d'infos
    \item Apres la fin de la boucle, insert la fonction d'analyse
  \end{itemize}
\end{frame}


\begin{frame}
  \frametitle{Avant/Apres}
  \begin{columns}
    \begin{column}{0.45\textwidth}
      \begin{algorithm}[H]
        \begin{algorithmic}
          \FOR{i = 0 : N}
          \STATE{B[i] = A[i]}
          \ENDFOR
        \end{algorithmic}
      \end{algorithm}
    \end{column}

    \begin{column}{0.45\textwidth}
      \begin{algorithm}[H]
        \begin{algorithmic}
          \FOR{i = 0 : N}
          \STATE{Initialize info struct}
          \STATE{Store access info - A[i]}
          \STATE{Store access info - B[i]}
          \STATE{B[i] = A[i]}
          \ENDFOR
          \STATE{Analysis}
          \STATE{Cleanup}
        \end{algorithmic}
      \end{algorithm}
    \end{column}
  \end{columns}
\end{frame}

\begin{frame}
  \frametitle{Bibliotheque Externe - Stockage}
  \begin{block}
    {access\_t}
    void *start\\void *end\\enum io\_type type
  \end{block}
  
  \begin{block}
    {vector$\langle$access$\rangle$ *current\_loop}
  \end{block}

  \begin{block}
    {vector$\langle$*vector$\langle$access$\rangle\rangle$ full\_loop}
  \end{block}
  
  \begin{itemize}
    \item Debut de chaque boucle: alloue current\_loop, ajoute a full\_loop
    \item Dans la boucle: pour chaque acces, remplis un access\_t, ajoute a current\_loop
    \item Apres la boucle: analyse, vide full\_loop
  \end{itemize}
\end{frame}





\end{document}


